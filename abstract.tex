\documentclass[10pt,oneside]{article}
\usepackage[a4paper,top=1in,bottom=0.5in,left=0.5in,right=0.5in]{geometry}
\usepackage{setspace}\onehalfspacing
\usepackage{fancyhdr}
\usepackage{enumerate}
\fancyhead[l]{DOMINGO, Kenneth V. \\ BS Applied Physics (Instrumentation)}
\pagenumbering{gobble}
\pagestyle{fancy}
\parindent 0pt
\parskip=0.25em
\pretolerance=5000
\tolerance=9000
\emergencystretch=0pt
\righthyphenmin=4
\lefthyphenmin=4

\begin{document}

\paragraph{Tentative Thesis Title} Compressive Sensing: Applications from 1-D to $N$-D

\subsubsection*{Physics & Astronomy Classification Scheme}
\begin{itemize}
	\item 43.60.Vx \quad Acoustic sensing and acquisition
	\item 07.05.Mh \quad Neural networks, fuzzy logic, artificial intelligence
	\item 84.40.Ua \quad Telecommunications: signal transmission and processing; communication satellites
\end{itemize}

\subsubsection*{Problem Statement}
I present a generalization of compressive sensing to signals of arbitrary dimensions for applications such as compression, encryption, and enhancement.

\subsubsection*{Significance}
Modern signal acquisition technologies are made possible by the Nyquist-Shannon sampling theorem (NST) which states that all the pertinent signal information can be acquired by sampling it at a rate at least twice the signal's highest frequency. However, this process is extremely wasteful as the signal is compressed before storing it by systematically discarding the imperceptible information. Compressive sensing (CS) aims to directly sense the information that survives the compression stage. Current literature focus exclusively on either audio or image signals. This study aims to be able to process signals of arbitrary dimensions such as those consisting of combinations of audio and images.

\subsubsection*{Methodology}
\begin{enumerate}
	\item Prepare the data to be compressively sensed which includes images, speech and video.
	\item Two general CS workflows arise from processing of images and audio. Higher dimensional signals can be split into factors of these two and processed with the appropriate workflow.
	\item Identify appropriate quantitative metrics for each type of signal which are relatively intuitive when compared to perceptual metrics, and identify what factors in the signal they are dependent on.
\end{enumerate}

\subsubsection*{Major References}
\begin{enumerate}[(1)]
	\item E. J. Cand\`{e}s, J. K. Romberg, and T. Tao, Stable signal recovery from incomplete and inaccurate measurements, \textit{Commun. Pure Appl. Math.} \textbf{59}, 1207 (2006), arXiv:0503066v2.
	\item M. Iliadis, L. Spinoulas, and A. K. Katsaggelos, Deep fully-connected networks for video compressive sensing, \textit{Digit. Signal Process. A Rev. J.} \textbf{72}, 9 (2018).
	\item S. Y. Low, D. S. Pham, and S. Venkatesh, Compressive speech enhancement, \textit{Speech Commun.} \textbf{55}, 757 (2013).
\end{enumerate}

\medskip

Submitted by: \\
\smallskip \\
Kenneth V. Domingo \\

Endorsed by: \\
\smallskip \\
Maricor N. Soriano \\
Thesis Adviser

\end{document}