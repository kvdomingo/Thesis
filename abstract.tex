\chapter*{Abstract}

Modern signal acquisition technologies are made possible by the Nyquist-Shannon sampling theorem. However, this paradigm is extremely wasteful, as the signal is compressed before storage by systematically discarding the imperceptible information. Compressive sensing (CS) is an alternative sampling paradigm which aims to directly sample the relevant information, which would not otherwise be thrown away through conventional means. Current literature focus either on formulating more computationally-efficient reconstruction algorithms, memory-efficient sensing matrices, or methods which improve CS reconstruction quality. In this thesis, I present a standardized, unified CS workflow for signals of arbitrary dimensions. Additionally, I quantify the reconstruction quality of compressively sampled signals with perceptually intuitive metrics: the structural similarity index (SSIM) for image-based signals, and the perceptual evaluation of speech quality (PESQ) for audio-based signals. I show that with this workflow, CS can be applied to signals of arbitrary content for applications such as compression, encryption and enhancement.