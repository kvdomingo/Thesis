\chapter{Introduction}
\label{chap:intro}

This study explores the use of compressive sensing (CS)\replaced{,}{---} an emergent sampling theorem that allows reconstruction of signals from much fewer samples than required by the Nyquist-Shannon sampling theorem (NST)\deleted{---as a viable method for compression, encryption, and/or enhancement}. In this framework, the computational burden of encoding/decoding a signal is shifted from the sampling device to the device performing reconstruction, decompression, or other modes of post-processing. As such, there exist many ways to reconstruct a signal from compressive measurements.

CS has found its applications in simple audio signals containing stable frequencies (such as pure tones \cite{Mathew2016,Andras2018}) and dynamic frequencies (such as speech \cite{Low2013,Low2018,Abrol2015}), images \cite{Mo2013,Zhou2016,Romero2016}, and grayscale videos \cite{Liu2014,Chen2014}. The formulation of a sensing matrix in CS requires a basis conforming to some uniform uncertainty principle, and most common starting points would be partial discrete cosine transform (DCT) matrices or partial discrete wavelet transform (DWT) matrices. \deleted{Recent studies, however, have shown that learned bases perform much better on more complex signals \cite{Liu2013,Sharma2018,Eslahi2016}, i.e., those that would be typically encountered in real life situations. The learning algorithms associated with the construction of these bases range from classical iterative methods, which have long been used in optimization problems, to the more contemporary machine learning methods.}


\section{Related literature}
\label{sec:rrl}
In 2004, Cand\`{e}s, Romberg, Tao \cite{Candes2006}, and Donoho \cite{Donoho2006} asked the question, 

\begin{quote}
	With the recent breakthroughs in lossy compression technologies, we now know that most of the data we acquire can be thrown away with minimal perceptual loss. Why bother to acquire all the data when we can just directly measure the part that will not be thrown away?
\end{quote}

\noindent This was eventually answered in many different ways by the same people, ultimately birthing the field which we now know as compressive sensing. The methods in CS apply concepts from time-frequency uncertainty principles \cite{Donoho2001} and sparse representations, which were studied rigorously by Donoho and Elad \cite{Donoho2003}. CS can be viewed as a strategic undersampling method: the signal is sampled at random \replaced{points}{locations in the real domain}, and the ratio of the indices where it is sampled to the size of the signal can be associated with some quasi-frequency which may be lower than the Nyquist rate.

\added{Linh-Trung et al.} \cite{LinhTrung2008} demonstrated the use of deterministic chaos filters to acquire samples instead of random distributions. Sampling using a Gaussian-Logistic map was applied to acoustic signals in \cite{Mathew2016}. Normally, a deterministic chaotic function will need one or more initialization values as a ``seed'', and the sequence of numbers produced by different combinations of initial values rapidly diverge from each other. This phenomenon led to investigating the use of compressive sensing as an encryption algorithm. Simultaneous compression and encryption was achieved by \cite{Mo2013}, and it was found that the initial values were sensitive to perturbations on the order of $10^{-15}$. Their image compression-encryption model via compressive sensing was shown to have a key space on the order of $10^{34}$, making it extremely resistant to brute force and other types of attacks.\deleted{This was extended in \cite{Zhou2016} to utilize a higher-dimensional variant of the Lorenz attractor, subsequently expanding the key space to the order of $10^{83}$.} In the methods above, sampling was performed in the signal domain (i.e., temporal domain for audio, spatial domain for images), and the reconstruction was performed in the frequency domain with a DCT or similar basis. \added{A method was proposed by} \cite{Andras2018} \deleted{proposed a method} to perform both sampling and reconstruction in the time domain using differential evolution.

Audio signals, compared to images, \replaced{have higher information density}{are much more densely packed with information}. Whereas images are not naturally bandlimited and rather, are dependent on the spatial resolution and bit depth of the imaging device, audio size scales proportionally with time and takes on a wider range of values. The accepted frequency range of human hearing is from 20~Hz to 20~kHz, so by the NST, a sampling frequency \replaced{greater than}{at least} 40~kHz is needed to ensure that an audio sample is recorded \replaced{completely and tailored for humans}{correctly}. Any meaningful audio recording, especially those containing speech, will certainly have a duration of \replaced{much more than a few seconds}{a few seconds up to a few hours}, so one cannot straightforwardly apply methodologies used for images or recordings with relatively static frequencies, as the first challenge this would pose for electronic systems is insufficient memory to process the entire signal all at once. Low \cite{Low2013,Low2018} circumvented this problem by transforming the signal to the modulation domain, \deleted{i.e., the signal's spectrogram,} essentially raising a one-dimensional signal to $N$-dimensions, where the value of $N$ is dependent on the desired spectrogram resolution, number of subbands, and percent overlap between adjacent subbands. In such signals, recordings with an observed noise floor could be easily be denoised, which is an inherent property of CS \cite{Dabov2007}.

Due to the large size of video information as a consequence of its high dimensionality, it is possible, but impractical, to apply image CS techniques on an entire frame-by-frame basis. Correlations between adjacent frames are utilized instead, and can be obtained using dictionary learning \cite{Liu2013} or principal components analysis \cite{Liu2014}. For the same reason, the application of CS to grayscale videos presuppose the use of machine learning methods. Iliadis \cite{Iliadis2018,Iliadis2020} came up with two different deep neural network architectures whose inputs and outputs are patches derived from grayscale videos. This idea was utilized in \cite{Yao2019} who modified the architecture into a residual network containing several convolutional layers. The original design was targeted towards image reconstruction, but could easily be extended to videos.

\deleted{In the same vein, neural network methods could also be used in CS of speech. Advances in natural language processing were primarily made using recurrent neural networks (RNN). In \cite{Xia2015}, a speech signal was first modeled by their proposed RNN architecture based on a noise-constrained least squares estimate, and final recovery is done via Kalman filtering. A new simple recurrent unit (SRU) network was created in \cite{Cui2020} which maps the relation between noisy and clean speech recordings for speech enhancement.}


\section{Novelty}
\label{sec:novel}
This study aims to provide a generalization for applying CS techniques to signals of arbitrary dimensions\added{, for applications such as compression, encryption, and enhancement}. \replaced{Contemporary}{Previous} CS research worked exclusively either audio or image \replaced{signals}{sequences}\deleted{as the target for CS}, and due to the computational demands, \replaced{and}{the} focus\added{es} \deleted{of most of the research in the field has been to} \added{constructing effective sensing matrices,} optimiz\replaced{ing}{e} the computational complexity for real-time applications, and improving \deleted{signal} reconstruction quality. In the establishment of CS methods, two different general frameworks \deleted{to compressively sample signals} arise\deleted{, namely, one-dimensional CS (1DCS) and two-dimensional CS (2DCS)} \added{for image and audio signals}. \deleted{It is shown that an $N$-dimensional signal can be decomposed into factors of one-dimensional and two-dimensional signals, and can be processed using methods appropriate for each type of signal. Furthermore, it is shown that $N$ is bound not only by the type of signals being worked with, but also the computational power of the decoding/decompressing device. In particular, large values of $N$ are useful in encryption, where a signal is first raised to a high dimension in a certain basis, the sensing matrix is derived from another high-dimensional basis, and the result is cast back to either one or two dimensions to yield the encrypted message.}

\added{Furthermore, current research tend to evaluate signal reconstruction quality using statistical metrics, such as mean-squared error (MSE) and its variants. Arguably, the final interpreter of all signals are humans, and it is important to be able to tell how well any compressive algorithm performs just by looking at the metrics without directly observing the signal contents. In light of this, the study also aims to evaluate the reconstruction quality of CS algorithms using perceptually accurate metrics. This class of objective metrics are usually built upon now-obsolete subjective scoring systems, and allows human observers to make an informed estimate of the signal quality without directly accessing the signal itself.}

\added{Finally, this study also aims to lay out a unified, standardized workflow for similar applications of CS on signals with arbitrary content. This includes signals containing a combination of audio and images, such as color videos and hyperspectral images.}


\section{Thesis overview}
\label{sec:overview}
The next chapter establishes the relevant mathematical concepts and notation to be used throughout this study, algorithms used in signal reconstruction, and appropriate metrics per type of signal. \added{Chapter~3 establishes basic workflows and studies the effect of random sampling on CS reconstruction.} Chapters~\replaced{4 \& 5}{3--5} respectively focus on \replaced{image-based CS and audio-based CS}{two-dimensional CS, one-dimensional CS, and $N$-dimensional CS}. Each of these chapters are self-contained methodologies, results, and discussions to emphasize that the methods can work independently of each other. Conclusions of the study and recommendations for future studies are presented in Chapter~6.