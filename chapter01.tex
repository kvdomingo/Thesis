The trend of both curiosity and profit-driven human development has caused a surge in the amount of openly accessible raw data. More often than not, the data is generated much faster than it can be processed. In the endeavor of keeping up with information inflow, there are two major factors that significantly hinder our progress. First, Moore's law implicitly sets a physical limit to the number of transistors that can be placed on a fixed-size chip (barring a paradigm change in the design and manufacture of semiconductors), consequentially limiting how powerful and how fast electronic systems can become. The second is the Nyquist-Shannon sampling theorem (NST), which limits the range of frequencies a recording device can successfully capture. This study explores the use of compressive sensing (CS)---an emergent sampling theorem that allows reconstruction of signals from much fewer samples than required by the NST by taking advantage of some inherent sparsity or periodicity in a signal---as a viable method for compression, encryption, and/or enhancement. In this framework, the computational burden of encoding/decoding is shifted from the sampling device to the device performing reconstruction, decompression, and other modes of post-processing. As such, there exist many ways to reconstruct a signal from compressive measurements. CS has found its applications in simple audio signals containing pure tones \cite{Mathew2016,Andras2018} and speech \cite{Low2013,Low2018,Abrol2015}, images \cite{Mo2013,Zhou2016,Romero2016}, and audioless grayscale videos \cite{Liu2014,Chen2014}. The formulation of a sensing matrix in CS requires a basis conforming to some uniform uncertainty principle, and the most common starting point would be a partial discrete cosine transform (DCT). Recent studies, however, have shown that learned bases perform much better on more complex signals \cite{Liu2013,Sharma2018,Eslahi2016}, i.e., those that would be typically encountered in practical situations. The learning algorithms associated in the construction of these bases range from classical iterative methods, which have long been used in optimization problems, to the more contemporary machine learning methods.


\section{Related literature}
\label{sec:rrl}
In 2006, Cand\`{e}s, Romberg, Tao \cite{Candes2006}, and Donoho \cite{Donoho2006} asked the question, 

\begin{quote}
	With the recent breakthroughs in lossy compression technologies, we now know that most of the data we acquire can be thrown away with minimal perceptual loss. Why bother to acquire all the data when we can just directly measure the part that will not be thrown away?
\end{quote}

\noindent which was eventually answered in many different ways by the same people, ultimately birthing the field which we now know as compressive sensing. The methods in CS apply concepts from time-frequency uncertainty principles \cite{Donoho2001} and sparse representations, which were studied rigorously by Donoho and Elad \cite{Donoho2003}. CS can be viewed as a strategic undersampling method: the signal is sampled at random locations in the real domain, and the ratio of the indices where it is sampled to the signal size can be associated with some quasi-frequency which may or may not be lower than the Nyquist rate. The reason for this indifference towards the Nyquist rate is that the compressive samples are not equitemporal and thus, cannot ``violate'' the Nyquist theorem. Therefore, CS should not be considered a ``replacement'' or ``rival'' to the NST, but rather, a completely distinct model of signal acquisition.

\cite{LinhTrung2008} demonstrated the use of deterministic chaos filters to acquire samples instead of random distributions. Sampling using a Gaussian-Logistic map was applied to acoustic signals in \cite{Mathew2016}. Normally, a deterministic chaotic function will need one or more initialization values as a ``seed'', and the sequence of numbers produced by different combinations of initial values rapidly diverge from each other. This phenomenon led to investigating the use of compressive sensing as an encryption algorithm. Simultaneous compression and encryption was achieved by \cite{Mo2013}, and it was found that perturbations on the order of $10^{-15}$ to the initial values are sufficient to make an encrypted image remain as such. Their image compression-encryption model via compressive sensing was shown to have a key space on the order of $10^{34}$, making it extremely resistant to brute force and other types of attacks. This was extended in \cite{Zhou2016} to utilize higher-dimensional chaotic systems, subsequently expanding the key space to the order of $10^{83}$. In the methods above, sampling was performed in the signal domain (i.e., time domain for audio, spatial domain for images), and the reconstruction was performed in the frequency domain. \cite{Andras2018} proposed a method to perform both sampling and reconstruction in the time domain using differential evolution.

Audio signals, compared to images, are much more densely packed with information. Whereas image size is dependent on the resolution and bit depth of the imaging device, audio size scales proportionally with time. The accepted frequency range of human hearing is from 20 Hz to 20 kHz, so by the NST, a sampling frequency of at least 40 kHz is needed to preserve the fidelity of recorded audio. Any meaningful audio recording, especially those containing speech, will certainly have a duration of a few seconds up to a few hours, so one cannot straightforwardly apply methodologies used for images or recordings with relatively static frequencies, as the first challenge this would pose for electronic systems is insufficient memory to process the entire signal all at once. Low \cite{Low2013,Low2018} circumvented this problem by transforming the signal to the modulation domain, i.e., the signal's spectrogram, essentially raising a one-dimensional signal to $N$-dimensions, where the value of $N$ is dependent on the desired spectrogram resolution, number of subbands, and percent overlap between adjacent bands. In such signals, recordings were also observed to have a noise floor, which could be easily removed by an inherent denoising property of CS \cite{Dabov2007}.

Due to the large size of video information as a consequence of its high dimensionality, it is possible, but impractical, to apply image CS techniques on an entire frame-by-frame basis. Correlations between adjacent frames are utilized instead, and can be obtained using dictionary learning \cite{Liu2013} or principal components analysis \cite{Liu2014}. For the same reason, the application of CS to grayscale videos presuppose the use of machine learning methods. Iliadis \cite{Iliadis2018,Iliadis2020} came up with two different deep neural network architectures whose inputs and outputs are patches derived from grayscale videos.


\section{Novelty}
\label{sec:novel}
Previous studies worked with either pure audio or pure image/image sequence signals as the target for compressive sensing. This study works on applying previously established compressive sensing methods to $N$-dimensional signals, such as spectrograms and full-color video coupled with audio. $N$ will be bound by the type of signals being worked with, as well as the computational power of the decoding device. In particular, large values of $N$ are useful in encryption, where a signal is first raised to a high dimension, the sensing matrix is derived also from a high dimension, and the result is usually cast back to dimension one or two.