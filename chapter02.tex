The trend of both curiosity and profit-driven human development has caused a
surge in the amount of openly accessible raw data. More often than not, the data is generated much faster than it can be processed into something interpretable or useful. In the endeavor of keeping up with the inflow of information, there are two major factors that significantly hinder our progress. First, Moore’s law implicitly sets a physical limit to the number of transistors that can be placed on a chip, consequently limiting how powerful and how fast electronic systems can become (barring a paradigm shift in the fundamental design of semiconductors). The second is the Nyquist-Shannon sampling theorem (NST), which limits the range of frequencies a recording device can successfully capture. This states that given that you know a signal's highest frequency component $f_B$, sampling it at a rate $f_S$ that is at least twice this frequency is sufficient to capture all of the pertinent information regarding that signal; that is $f_S \geq 2f_B$, where $f_B$ is known as the Nyquist rate \cite{Shannon1949}, and also as the signal bandwidth. For signals that are not naturally bandlimited, such as images, the ability reproduce a signal is dependent on the device's resolution and still follows the same principle: there should be at least twice the number of pixels in the same dimension as the image's highest spatial frequency. For practical day-to-day use, the NST covers everything you need. However, issues arise when bandwidth and storage are at a premium. Typically, after sensing a signal, not all of the raw data is stored. Rather, this data is converted to a compressed format by systematically discarding values such that the loss of information is virtually imperceptible. Thus, the process of acquiring massive amounts of data followed by compression is extremely wasteful \cite{Candes2008b}. Enter compressive sensing.

\section{Compressive sensing}
\label{sec:cs}

Consider a signal $\vec{x} \in \mathbb{C}^{N \times 1}$. This notation indicates that $\vec{x}$ is complex-valued and is one-dimensional with cardinality $N$. The process of acquisition or sensing this signal can be modeled as a linear system, where the physical signal properties we wish to capture are transformed into digital values by correlating them with a waveform basis as

\begin{equation}\label{eq:cesa}
    y_k = \innerproduct{\vec{x}}{\varphi_k} \quad , \quad k \in \mathbb{Z}
\end{equation}

\noindent In the sufficiently sampled NST case, $\varphi_k$ are Dirac basis vectors which turn $\vec{y}$ into a vector containing samples of $\vec{x}$ in the signal domain (i.e., temporal domain for audio, spatial domain for images).