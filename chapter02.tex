\section{Compressive sensing}
\label{sec:cs}

Nearly all of today's signal acquisition technologies were made possible thanks to the Nyquist-Shannon sampling theorem (NST). This states that given that you know a signal's highest frequency component $f_B$, sampling it at a rate $f_S$ that is at least twice this frequency is sufficient to capture all of the pertinent information regarding that signal; that is $f_S \geq 2f_B$, where $f_B$ is known as the Nyquist rate \cite{Shannon1949}, and also as the signal bandwidth. For signals that are not naturally bandlimited, such as images, the ability reproduce a signal is dependent on the device's resolution and still follows the same principle: there should be at least twice the number of pixels in the same dimension as the image's highest spatial frequency. For practical day-to-day use, the NST covers everything you need. However, issues arise when bandwidth and storage are at a premium. Typically, after sensing a signal, not all of the raw data is stored. Rather, this data is converted to a compressed format by systematically discarding the imperceptible information. Thus, the process of acquiring massive amounts of data followed by compression is extremely wasteful \cite{Candes2008b}, and so the goal of CS is to directly acquire the information which survives the compression stage.

Consider a real-valued signal $\vec{x} \in \mathbb{R}^{N \times 1}$. The process of acquisition or sensing this signal can be modeled as a linear system, where the physical signal values are transformed into values that can be understood by some sensing device by applying a linear transformation as

\begin{equation}\label{eq:cesa}
    \vec{y} = \vec{A}\vec{x}
\end{equation}

\noindent where $\vec{A} \in \mathbb{R}^{M \times N}$ is the transformation commonly known as a sensing matrix. In the NST case, $\vec{A}$ is a square matrix ($M = N$), and one could easily reconstruct $\vec{x}$ by inversion. However, we are interested in the case when the signal is undersampled, i.e., the number of measurements is less than the size of the signal ($M \ll N$).